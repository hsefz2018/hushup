\documentclass[UTF8, 11pt, a4paper]{article}
\usepackage[cm]{sfmath}
\usepackage{tabularx}
\def\arraystretch{1.3}
\usepackage[a4paper, top=3.18cm,bottom=3.81cm,left=2.54cm,right=2.54cm]{geometry}
\usepackage{indentfirst}
\setlength{\parskip}{6pt}
\XeTeXlinebreaklocale "zh"
\usepackage{graphicx}
\usepackage[normalem]{ulem}

\usepackage{fontspec}
\setmainfont{思源黑体}
\SetSymbolFont{largesymbols}{normal}{OMX}{iwona}{m}{n}
\setmonofont{Source Code Pro}

\begin{document}
\section*{勇攀高峰 / Challenge Accepted \makebox[2.5em]{} \small{「CACCEPT」}}

Kurumi 是一位资深 osu!mania 玩家。%
当被问起滋瓷哪种键位的时候,Kurumi 总是答道:“我可以跟你们说一句无可奉告。”%

“但是你们又不高兴,我怎么办?”又一次被问到“认为 SDFJK 键位吼不吼啊”的问题之后,%
Kurumi 表示无奈,“问出来的问题啊,都 too simple!你们这样子是不行的!”为了摆脱%
这些无止境的问题,Kurumi 决定告诉新闻工作者们一些人生的经验。

Kurumi 最滋瓷的键位排列是一个五个不同字母的排列 $K_{1 \dots 5}$。%
但是 Kurumi 是身经百战了,见得多了,决定让新闻工作者们猜出这个键位。%
对于每一次猜测,Kurumi 都只会给出非常少量的信息,所以 Kurumi 今天算是得罪了他们一下。

\subsection*{任务}
对于一个未知的五个\uwave{互不相同}的字母排列 $K$,%
通过一系列询问确定 $K$ 中五个字母的排列。

\subsection*{实现}
在本题中,你不必编写一个完整的程序,而只需要实现一个过程 \texttt{newGame()}%
用以进行一组 $K$ 的交互并给出答案。该方法不接收任何参数。

\texttt{newGame()} 可以多次调用函数 \texttt{makeAttempt(a, b, c, d, e)}%
来进行一次猜测。该函数接收五个字符类型的参数 \texttt{a, b, c, d, e}%
(\uwave{不必互不相同}),%
并返回一个整数 $R = 10 R_1 + R_2 (0 \leq R_1, R_2 \leq 5)$,其中:
\begin{itemize}
    \item $R_1$ 表示五个位置中有多少个位置上的字母和 $K$ 中相同(不区分大小写,下同);
    \item $R_2$ 表示除去正确的位置上的字母,在本次猜测中还有多少个%
        \uwave{不同的}字母在 $K$ 中出现过。
\end{itemize}

\texttt{newGame()} 需要通过一次或多次对 \texttt{makeAttempt(a, b, c, d, e)}%
的调用来确定 $K$ 中五个字母的排列。当你的程序能够确定这个答案时,把 $K$ 作为参数%
调用一次 \texttt{makeAttempt},即 \texttt{makeAttempt(K\_1, K\_2, K\_3, K\_4, K\_5)},然后退出过程。

\subsection*{样例}
考虑 \texttt{K = "sdfjk"} 的栗子。

首先 \texttt{newGame()} 被调用。它进行的工作如下:
\begin{itemize}
    \item 调用 \texttt{makeAttempt('s', 'd', 'j', 'k', 'l')},得到返回值 22;
    \item 调用 \texttt{makeAttempt('d', 's', 'j', 'k', 'l')},得到返回值 4;
    \item 调用 \texttt{makeAttempt('s', 'd', 'f', 'o', 'p')},得到返回值 30;
    \item 调用 \texttt{makeAttempt('s', 'd', 'f', 'k', 'p')},得到返回值 31;
    \item 调用 \texttt{makeAttempt('s', 'd', 'f', 'j', 'k')},得到返回值 50 并退出。
\end{itemize}

\subsection*{评分}
评分时,本题共有 20 个预先确定的不同的 $K$,\texttt{newGame()} 将\uwave{在同一次程序运行中}被调用恰好 20 次。%
(因此你可以使用预处理、静态/全局变量等方式来加速程序。)

若程序发生运行时错误或运行超过时限,本次提交得分为 0;否则得分按照下面的方式计算:

$$
    \mathrm{TotalScore} = \lfloor A + E \rfloor
$$

其中:
\begin{itemize}
    \item \textbf{准确度得分} $A$:等于在 20 组数据中获得正确答案的个数。
    \item \textbf{效率得分} $E$:
    \begin{itemize}
        \item 若 $A \neq 20$:$E = 0$;
        \item 若 $A = 20$:
            陪审团设定了一个猜测次数的下界 $L$,但 $L$ 的具体数值在比赛期间不公布。
            假设你的程序在第 $i$ 组数据上调用了 $C_i$ 次 \texttt{makeAttempt} 函数,那么

            $$
                E = 4 \cdot \sum_{i = 1}^{20} \min\left(1, \frac{L}{C_i}\right)
            $$
    \end{itemize}
\end{itemize}

\subsection*{细节 \& 本地测试}
在本题的附件中包含 \texttt{grader.[c|cpp|pas]},\texttt{caccept.[c|cpp|pas]},%
\texttt{compile\_[c|cpp|pas].bat} 以及一些附加文件。

其中 \texttt{grader.[c|cpp|pas]} 是用于在本地测试程序的函数库,\texttt{caccept.[c|cpp|pas]} %
分别是对应语言的样例程序,实现算法时直接在其基础上进行修改即可。提交评测时只需提交该文件。%
修改完毕后,运行 \texttt{compile\_[c|cpp|pas].bat} 来编译你的程序。编译好的程序将从文件%
\texttt{caccept.in} 读取数据,调用 \texttt{newGame()},并输出对应的运行结果(正确性和调用次数)。%
你可以编辑 \texttt{caccept.in} 以测试算法在不同 $K$ 取值上的表现。

输入文件 \texttt{caccept.in} 不能超过 131072 行,%
每行包含五个字母(大小写均可)表示一组数据的 $K$。%
对于文件中的每一组数据,\texttt{newGame()} 将分别被调用一次。

\subsection*{限制}
\begin{itemize}
\item 时间:20.0 秒(所有 20 组数据之和)
\item 内存:1.0 GiB
\end{itemize}

\begin{figure}[h]\centering
\includegraphics[scale=0.55]{challengeaccepted.png}
\end{figure}

\end{document}

